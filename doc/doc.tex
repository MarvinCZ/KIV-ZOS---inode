\documentclass[a4paper,12pt]{article}
\usepackage[T1]{fontenc}
\usepackage[utf8]{inputenc}
\usepackage{lmodern}
\usepackage{graphicx}
\usepackage{microtype}
\usepackage[czech]{babel}
\usepackage{cite}
\usepackage{algorithm}
\usepackage{algpseudocode}
\usepackage{siunitx}
\usepackage{float}
\usepackage{mathtools}
\usepackage{svg}
\usepackage{listings}
\usepackage{titlesec}
\usepackage{textcomp}
\usepackage{xcolor}
\usepackage[hidelinks]{hyperref}
\usepackage[a4paper, total={6in, 8in}]{geometry}

\definecolor{backcolour}{rgb}{0.95,0.95,0.92}
\lstloadlanguages{Ruby}
\lstset{%
basicstyle=\ttfamily\color{black},
commentstyle = \ttfamily\color{red},
keywordstyle=\ttfamily\color{blue},
stringstyle=\color{orange},
captionpos=b,
backgroundcolor=\color{backcolour},
breaklines=true,
}
\renewcommand{\lstlistingname}{Zdrojový kód}
\renewcommand{\lstlistlistingname}{Seznam zdrojových kódů}

\floatstyle{boxed}
\restylefloat{figure}
\clubpenalty 1000
\widowpenalty 1000

\titleclass{\subsubsubsection}{straight}[\subsection]

\newcounter{subsubsubsection}[subsubsection]
\renewcommand\thesubsubsubsection{\thesubsubsection.\arabic{subsubsubsection}}
\renewcommand\theparagraph{\thesubsubsubsection.\arabic{paragraph}} % optional; useful if paragraphs are to be numbered

\titleformat{\subsubsubsection}
{\normalfont\normalsize\bfseries}{\thesubsubsubsection}{1em}{}
\titlespacing*{\subsubsubsection}
{0pt}{3.25ex plus 1ex minus .2ex}{0.0ex plus .2ex}

\titlespacing*{\section}
{0pt}{3.25ex plus 1ex minus .2ex}{0.0ex plus .2ex}
\titlespacing*{\subsection}
{0pt}{3.25ex plus 1ex minus .2ex}{0.0ex plus .2ex}
\titlespacing*{\subsubsection}
{0pt}{3.25ex plus 1ex minus .2ex}{0.0ex plus .2ex}

\makeatletter
\renewcommand\paragraph{\@startsection{paragraph}{5}{\z@}%
{3.25ex \@plus1ex \@minus.2ex}%
{-1em}%
{\normalfont\normalsize\bfseries}}
\renewcommand\subparagraph{\@startsection{subparagraph}{6}{\parindent}%
{3.25ex \@plus1ex \@minus .2ex}%
{-1em}%
{\normalfont\normalsize\bfseries}}
\def\toclevel@subsubsubsection{4}
\def\toclevel@paragraph{5}
\def\toclevel@paragraph{6}
\def\l@subsubsubsection{\@dottedtocline{4}{7em}{4em}}
\def\l@paragraph{\@dottedtocline{5}{10em}{5em}}
\def\l@subparagraph{\@dottedtocline{6}{14em}{6em}}
\makeatother

\let\olditemize=\itemize \let\endolditemize=\enditemize \renewenvironment{itemize}{\olditemize \itemsep0em}{\endolditemize}
\setlength{\parindent}{0cm}
\setlength{\parskip}{0.3cm}

\begin{document}

	\begin{titlepage}
		\centering

		\includegraphics[width=8cm]{favlogo.jpg}

		\vspace{2cm}

		{\Huge KIV/ZOS}

		\vspace{1cm}

		{\Large Zjednodušený filesystem založený na i-uzlech}

		\vspace{3cm}

		{\Large Martin Bruna (A20B0066P)}\par
		email: brunam@students.zcu.cz

		\vfill

		\today
	\end{titlepage}

	\setcounter{secnumdepth}{4}
	\setcounter{tocdepth}{4}
	\tableofcontents
	\newpage

	\section{Zadání}
	Tématem semestrální práce bude práce se zjednodušeným souborovým systémem založeným na i-uzlech.

	Systém bude podporovat příkazy: cp, mv, rm,mkdir, rmdir, ls, cat, cd, pwd, info, incp, outcp, load, format, ln

	Předpokládáme korektní syntax zadávaných příkazů, ale ne sémanticky.
	Délka souboru je omezena na 8+3 znaků plus ukončující 12. byte.
	\newpage
	\section{Teorie}
	Souborov systém založený na i-nodech se zkládá z několika částí
	\subsection{Super blok}
	Hlavička systému, obsahuje počáteční adresy jednotlivých částí a další konfiguraci celého systému.
	\subsection{I-node a cluster bitmapa}
	Dvě sekce, která každá představuje jedno bitové pole označující, které části i-node/cluster sektoru jsou zaplněny/volné.
	\subsection{Sektor i-nodu}
	Sektor, kde se nachází všechna informační data i-nodu.
	\subsubsection{I-node}
	I-node je identifiktor a nositel informací pro jednotlivé soubory. Má na sobě flag zda jde o složku nebo běžný soubor. Zároveň obsahuje 5 direct linků (přímí odkaz na pamět v cluster sektoru), adresu na indirect cluster, který v sobě má seznam adres na reálná data a v poslední řadě odkaz na 2x nepřímí odkaz (odkaz na seznam oskazů na odkazy do reálných dat).

	Kromě toho má i-node na sobě uloženo, jaká je velikost dat a počet složek ve kterých se na daný i-node odkazuje.
	\subsection{Sektor clusterů}
	Sektor, kde se nachází data souborů.

	\newpage
	\section{Uživatelská dokumentace}
	Program byl testován s překladem pomocí cmake ve verzi 3.19.

	Jako parametr program přijímá cestu k souboru do kterého je/bude uložen celý filesystem. Pokud parametr není zadán je defaultně zvolen soubor fs.dat

	Před používáním nového filesystému je potřeba provést formátování pomocí příkazu format.

	Příkazy: viz. zadání
	
	\newpage
	\section{Programatorská dokumentace}
	Hlavním vstupem do programu je main.cpp, který spustí instanci Console
	\subsection{Console - Console.hpp + Console.cpp}
	Tvoří uživatelský interface aplikace a předává uživatelem zadané příkazy dál
	\subsection{System - System.hpp + System.cpp}
	Obsahuje veškerou vysokoúrovňovou logiku - poskytuje implementaci jednotlivých příkazů
	\subsection{Directory - Directory.hpp + Directory.cpp}
	Poskytuje možnost práce se složkamy
	\subsection{INode - INode.hpp + INode.cpp}
	Obalka logiky kolem samotného inodu, primárně poskytuje vstupní/vystupní stream.
	\subsection{MemoryIterator - MemoryIterator.hpp + MemoryIterator.cpp}
	Vlastní logika průchodu daty inodu, vytváření nocýh odkazů/mazání nepotřebných
	\subsection{FileSystem - FileSystem.hpp + FileSystem.cpp}
	Nejnižší úroveň - přístup k zapisování přímo na filesystem, řeší správu bitmap, formátování zápis a čtení z cluterů.
    
	\newpage
	\section{Závěr}
	Výsledkem práce je funkční zjednodučený i-node based filesystem. Systém je zpracovaný v C++. Práce zabrala přibližně 30 hodin.

\end{document}

